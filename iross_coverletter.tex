%%%%%%%%%%%%%%%%%%%%%%%%%%%%%%%%%%%%%%%%
% Plain Cover Letter
% LaTeX Template
%
% This template has been downloaded from:
% http://www.latextemplates.com
%
% Original author:
% Rensselaer Polytechnic Institute (http://www.rpi.edu/dept/arc/training/latex/resumes/)
%
%%%%%%%%%%%%%%%%%%%%%%%%%%%%%%%%%%%%%%%%%

%----------------------------------------------------------------------------------------
%	PACKAGES AND OTHER DOCUMENT CONFIGURATIONS
%----------------------------------------------------------------------------------------

\documentclass[11pt]{letter} % Default font size of the document, change to 10pt to fit more text

%\usepackage{newcent} % Default font is the New Century Schoolbook PostScript font 
%\usepackage{helvet} % Uncomment this (while commenting the above line) to use the Helvetica font
\usepackage{color}
\usepackage[left=0.75in,top=0.5in,right=0.75in,bottom=0.5in]{geometry} % Document margins
\usepackage{datetime}
\usepackage{lipsum}

% Margins
\topmargin=-1in % Moves the top of the document 1 inch above the default
\textheight=8.5in % Total height of the text on the page before text goes on to the next page, this can be increased in a longer letter
\oddsidemargin=-10pt % Position of the left margin, can be negative or positive if you want more or less room
\textwidth=6.5in % Total width of the text, increase this if the left margin was decreased and vice-versa


\def\today{
  \textcolor[rgb]{0.188,0.337,0.549}{\large{\bf 27.March.2013}}
  \smallskip
  \hrule
}

\def\recip{Mr. or Mrs. Awesomejob}
\def\recipaddr{Recruitment Office\\
    Awesome, Inc.\\
    123 Fake St.\\
Cool City, USA xxxxx}

\let\raggedleft\raggedright % Pushes the date (at the top) to the left, comment this line to have the date on the right

\begin{document}

%----------------------------------------------------------------------------------------
%	ADDRESSEE SECTION
%----------------------------------------------------------------------------------------

\begin{letter}{\recip\\
\recipaddr}

%----------------------------------------------------------------------------------------
%	YOUR NAME & ADDRESS SECTION
%----------------------------------------------------------------------------------------

\begin{center}
    {\Large\bf Ian A. Ross} % Your name
    \smallskip\break
    608.561.1729 -- ian.ackerman.ross@gmail.com
    \smallskip\break
    131 Langdon Street, Apt. 3; Madison, WI; 53703
\end{center} 
%\vfill

\signature{Ian Ross} % Your name for the signature at the bottom

%----------------------------------------------------------------------------------------
%	LETTER CONTENT SECTION
%----------------------------------------------------------------------------------------


\opening{Dear \recip,} 
 
%PARAGRAPH ONE: State the reason for the letter, name the position or type of work you are applying for and identify the source from which you learned of the opening (i.e. career development center, newspaper, employment service, personal contact).
\lipsum[1]

%PARAGRAPH TWO: Indicate why you are interested in the position, the company, its products, services - above all, stress what you can do for the employer. If you are a recent graduate, explain how your academic background makes you a qualified candidate for the position. If you have practical work experience, point out specific achievements or unique qualifications. Try not to repeat the same information the reader will find in the resume. Refer the reader to the enclosed resume or application which summarizes your qualifications, training, and experiences. The purpose of this section is to strengthen your resume by providing details which bring your experiences to life. 
\lipsum[17]
 
%PARAGRAPH THREE: Request a personal interview and indicate your flexibility as to the time and place. Repeat your phone number in the letter and offer assistance to help in a speedy response. For example, state that you will be in the city where the company is located on a certain date and would like to set up an interview. Alternatively, state that you will call on a certain date to set up an interview. End the letter by thanking the employer for taking time to consider your credentials. 
\lipsum[137]

\closing{Sincerely yours,}


%\encl{Curriculum vitae, employment form} % List your enclosed documents here, comment this out to get rid of the "encl:"

%----------------------------------------------------------------------------------------

\end{letter}

\end{document}
